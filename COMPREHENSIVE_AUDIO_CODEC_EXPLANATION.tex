\documentclass[12pt]{article}
\usepackage[utf8]{inputenc}
\usepackage{amsmath}
\usepackage{amsfonts}
\usepackage{amssymb}
\usepackage{graphicx}
\usepackage{geometry}
\usepackage{listings}
\usepackage{xcolor}
\usepackage{hyperref}
\usepackage{booktabs}
\usepackage{multirow}
\usepackage{array}
\usepackage{longtable}

\geometry{margin=1in}

\title{\textbf{Complete Beginner's Guide to Neural Audio Codecs: \\ Understanding Quantization Methods and Audio Processing}}
\author{Taka Khoo \\ Neural Audio Codec Project with Minh Bui}
\date{\today}

\begin{document}

\maketitle

\tableofcontents
\newpage

\section{What is a Neural Audio Codec? (In Super Simple Terms)}

\subsection{The Big Picture: Why We Need This}

Imagine you're trying to send a voice message to a friend, but your internet connection is terrible. You have two options:
\begin{itemize}
    \item \textbf{Option 1:} Send the full, high-quality audio (might take hours and cost a fortune)
    \item \textbf{Option 2:} Compress the audio to make it smaller and faster to send (but it might sound worse)
\end{itemize}

\textbf{The Problem:} How do we make audio files much smaller while keeping them sounding good?

\textbf{Neural Audio Codec Solution:} Use artificial intelligence to find the smartest way to compress audio, learning from millions of examples of what sounds good and what doesn't.

\subsection{The Analogy: Smart File Compression}

Think of it like packing a suitcase:
\begin{itemize}
    \item \textbf{Traditional compression:} Like folding clothes randomly - it works but isn't optimal
    \item \textbf{Neural compression:} Like having a professional packer who knows exactly how to fold and arrange everything to fit perfectly
\end{itemize}

\subsection{What We're Doing in This Project}

In our project with Minh Bui, we're focusing on one specific part of audio codecs: **quantization methods**. Think of quantization as the "packing strategy" - how we convert continuous audio signals into discrete, compressed representations.

\textbf{Our Goal:} Compare different quantization methods to see which one gives the best balance of:
\begin{itemize}
    \item \textbf{Quality:} How good the audio sounds
    \item \textbf{Compression:} How small the file becomes
    \item \textbf{Speed:} How fast we can compress/decompress
    \item \textbf{Stability:} How reliable the method is
\end{itemize}

\section{Understanding Audio Basics}

\subsection{What is Audio?}

Audio is just sound waves traveling through the air. When you speak, your vocal cords vibrate, creating pressure waves that travel to someone's ear.

\textbf{The Analogy:} Like throwing a rock in a pond:
\begin{itemize}
    \item \textbf{The rock hitting water} = your voice starting to speak
    \item \textbf{The ripples spreading out} = sound waves traveling through air
    \item \textbf{The ripples reaching the other side} = sound reaching someone's ear
\end{itemize}

\subsection{How Computers Store Audio}

Computers can't store continuous waves directly. Instead, they:
\begin{enumerate}
    \item \textbf{Sample the wave} thousands of times per second (like taking photos of the ripples)
    \item \textbf{Measure the height} of each sample (like measuring how high each ripple is)
    \item \textbf{Store these numbers} as a list (like keeping a list of all the measurements)
\end{enumerate}

\textbf{Example:} If you say "hello" for 1 second:
\begin{itemize}
    \item Computer takes 16,000 samples (16 kHz sampling rate)
    \item Each sample is a number between -32,768 and 32,767 (16-bit audio)
    \item Total storage: 16,000 × 2 bytes = 32,000 bytes = 32 KB
\end{itemize}

\subsection{Why Audio Files Are Big}

\textbf{The Problem:} High-quality audio needs lots of samples:
\begin{itemize}
    \item \textbf{CD Quality:} 44,100 samples per second × 2 bytes = 88.2 KB per second
    \textbf{High Quality:} 96,000 samples per second × 4 bytes = 384 KB per second
    \item \textbf{1 minute of music:} 384 KB × 60 = 23 MB
    \textbf{1 hour of music:} 23 MB × 60 = 1.38 GB
\end{itemize}

\textbf{The Solution:} Compression! Find ways to represent the same audio with fewer numbers.

\section{Understanding Quantization Methods}

\subsection{What is Quantization?}

Quantization is like rounding numbers to make them simpler. Instead of storing 3.14159265359..., we might store just 3.14.

\textbf{The Analogy:} Like measuring time:
\begin{itemize}
    \item \textbf{Exact time:} 2:37:42.156 seconds (very precise, needs lots of storage)
    \textbf{Quantized time:} 2:38 minutes (less precise, needs less storage)
\end{itemize}

\subsection{Why Quantization Matters}

\textbf{The Trade-off:}
\begin{itemize}
    \item \textbf{More quantization} = smaller files but worse quality
    \textbf{Less quantization} = better quality but bigger files
\end{itemize}

\textbf{The Challenge:} Find the sweet spot where files are small enough but quality is still good enough.

\section{Vector Quantization (VQ): The Simple Method}

\subsection{What is Vector Quantization?}

Vector Quantization is like creating a "dictionary" of common audio patterns and using codes to represent them.

\textbf{The Analogy:} Like using emojis instead of words:
\begin{itemize}
    \item \textbf{Instead of writing:} "I'm happy today!"
    \textbf{Use emoji:} 😊
    \item \textbf{Instead of storing:} 1000 audio samples
    \textbf{Use code:} "A7" (which means "happy sound pattern")
\end{itemize}

\subsection{How VQ Works}

\begin{enumerate}
    \item \textbf{Create a codebook:} Find common patterns in your audio data
    \item \textbf{Assign codes:} Give each pattern a unique number (like A1, A2, A3...)
    \item \textbf{Replace patterns:} Instead of storing the actual audio, store the code
    \item \textbf{Reconstruct:} When playing back, look up the code to get the original pattern
\end{enumerate}

\textbf{Example:} If you have 1 second of audio:
\begin{itemize}
    \item \textbf{Original:} 16,000 numbers (32 KB)
    \textbf{VQ compressed:} 100 codes (200 bytes)
    \textbf{Compression ratio:} 160:1 (160 times smaller!)
\end{itemize}

\subsection{The Mathematics of VQ}

\textbf{The Basic Idea:} Find the closest codebook entry to your input vector.

\textbf{Mathematical Formula:} For input vector $x$ and codebook vectors $\{c_1, c_2, ..., c_K\}$:
$$q(x) = \arg\min_{i} \|x - c_i\|^2$$

This means: "Find the codebook entry $c_i$ that's closest to our input $x$."

\textbf{What This Means:}
\begin{itemize}
    \item \texttt{argmin} = "find the index that gives the minimum value"
    \item $\|x - c_i\|^2$ = "squared distance between input and codebook entry"
    \item $q(x)$ = "the quantized version of our input"
\end{itemize}

\subsection{Strengths of VQ}

\begin{itemize}
    \item \textbf{Simple:} Easy to understand and implement
    \textbf{Fast:} Very quick to compress and decompress
    \item \textbf{Stable:} Always gives the same result
    \textbf{Memory efficient:} Only need to store the codebook
\end{itemize}

\subsection{Weaknesses of VQ}

\begin{itemize}
    \item \textbf{Limited expressivity:} Can only represent what's in the codebook
    \textbf{Codebook collapse:} Sometimes many codes end up representing the same thing
    \item \textbf{Poor gradient flow:} Hard to train with neural networks
    \textbf{Fixed bitrate:} Can't easily adjust compression level
\end{itemize}

\section{Residual Vector Quantization (RVQ): The Layered Method}

\subsection{What is Residual Vector Quantization?}

RVQ is like VQ but with multiple layers. Instead of trying to represent everything in one step, we do it progressively.

\textbf{The Analogy:} Like building a house:
\begin{itemize}
    \item \textbf{VQ approach:} Try to build the entire house perfectly in one step
    \textbf{RVQ approach:} 
    \begin{itemize}
        \item Layer 1: Build the foundation (basic structure)
        \item Layer 2: Add the walls (refine the structure)
        \item Layer 3: Add the roof (final details)
    \end{itemize}
\end{itemize}

\subsection{How RVQ Works}

\begin{enumerate}
    \item \textbf{First quantization:} Use VQ to approximate the audio signal
    \item \textbf{Calculate residual:} Find what's left over (the error)
    \item \textbf{Second quantization:} Use another VQ to approximate the residual
    \item \textbf{Repeat:} Keep going until the residual is small enough
\end{enumerate}

\textbf{Mathematical Formula:} For input $x$ and $L$ quantization stages:
\begin{align}
x_0 &= x \\
r_1 &= x_0 - Q_1(x_0) \\
r_2 &= r_1 - Q_2(r_1) \\
&\vdots \\
r_L &= r_{L-1} - Q_L(r_{L-1})
\end{align}

Where:
\begin{itemize}
    \item $x_0$ is the original input
    \item $r_i$ is the residual at stage $i$
    \item $Q_i$ is the quantization function at stage $i$
\end{itemize}

\textbf{The Final Output:} The sum of all quantized values:
$$x_{reconstructed} = Q_1(x_0) + Q_2(r_1) + Q_3(r_2) + ... + Q_L(r_{L-1})$$

\subsection{Why RVQ is Better Than VQ}

\begin{itemize}
    \item \textbf{More expressive:} Can represent more complex patterns
    \textbf{Progressive refinement:} Each layer adds more detail
    \item \textbf{Variable bitrate:} Can stop early for lower quality, continue for higher quality
    \textbf{Better quality:} Usually sounds better than single-stage VQ
\end{itemize}

\subsection{Why RVQ is More Complex}

\begin{itemize}
    \item \textbf{Slower:} Need to go through multiple stages
    \textbf{More memory:} Need to store multiple codebooks
    \item \textbf{Sequential:} Can't process stages in parallel
    \textbf{More parameters:} Harder to train and tune
\end{itemize}

\section{Product Quantization (PQ): The Parallel Method}

\subsection{What is Product Quantization?}

Product Quantization is like breaking a big problem into smaller, parallel problems.

\textbf{The Analogy:} Like organizing a big party:
\begin{itemize}
    \item \textbf{Traditional approach:} One person tries to organize everything
    \textbf{PQ approach:} Break it into committees:
    \begin{itemize}
        \item Food committee handles food
        \item Music committee handles music
        \item Decorations committee handles decorations
    \end{itemize}
\end{itemize}

\subsection{How PQ Works}

\begin{enumerate}
    \item \textbf{Split the vector:} Break your audio vector into subvectors
    \item \textbf{Quantize separately:} Apply VQ to each subvector independently
    \item \textbf{Combine results:} Put all the quantized subvectors back together
\end{enumerate}

\textbf{Mathematical Formula:} For input vector $x$ split into $M$ subvectors:
$$x = [x_1, x_2, ..., x_M]$$

Each subvector $x_i$ is quantized independently:
$$q_i(x_i) = \arg\min_{j} \|x_i - c_{i,j}\|^2$$

The final quantized vector is:
$$q(x) = [q_1(x_1), q_2(x_2), ..., q_M(x_M)]$$

\subsection{Why PQ is Fast}

\begin{itemize}
    \item \textbf{Parallel processing:} All subvectors can be quantized at the same time
    \textbf{Smaller codebooks:} Each subvector has its own, smaller codebook
    \item \textbf{Efficient search:} Finding the best match is faster in smaller spaces
    \textbf{Memory efficient:} Can use smaller, specialized codebooks
\end{itemize}

\subsection{Why PQ Might Be Worse}

\begin{itemize}
    \item \textbf{Loses correlations:} Might miss patterns that span across subvectors
    \textbf{Suboptimal partitioning:} The way we split the vector matters a lot
    \textbf{Not always perceptually good:} What's mathematically optimal might not sound good
\end{itemize}

\section{Gumbel-Softmax Quantization: The Differentiable Method}

\subsection{What is Gumbel-Softmax?}

Gumbel-Softmax is like VQ but "soft" - it can be trained with gradient descent, making it work better with neural networks.

\textbf{The Analogy:} Like learning to cook:
\begin{itemize}
    \item \textbf{VQ approach:} You have to choose exactly one recipe (hard decision)
    \textbf{Gumbel-Softmax approach:} You can blend multiple recipes together (soft decision)
\end{itemize}

\subsection{The Problem with VQ in Neural Networks}

\textbf{The Issue:} VQ makes "hard" decisions (choose exactly one codebook entry), but neural networks need "soft" decisions to calculate gradients.

\textbf{The Solution:} Gumbel-Softmax makes the quantization differentiable by adding randomness and temperature.

\subsection{How Gumbel-Softmax Works}

\begin{enumerate}
    \item \textbf{Calculate logits:} Find how similar your input is to each codebook entry
    \item \textbf{Add Gumbel noise:} Add random noise to make it stochastic
    \item \textbf{Apply softmax:} Convert to probabilities
    \item \textbf{Scale by temperature:} Control how "sharp" the distribution is
    \item \textbf{Sample:} Choose a codebook entry based on the probabilities
\end{enumerate}

\textbf{Mathematical Formula:} For input $x$ and codebook $\{c_1, c_2, ..., c_K\}$:

First, calculate similarities:
$$s_i = \text{similarity}(x, c_i)$$

Add Gumbel noise:
$$g_i = s_i + \epsilon_i \quad \text{where } \epsilon_i \sim \text{Gumbel}(0,1)$$

Apply softmax with temperature $\tau$:
$$p_i = \frac{\exp(g_i / \tau)}{\sum_{j=1}^K \exp(g_j / \tau)}$$

The quantized output is:
$$q(x) = \sum_{i=1}^K p_i \cdot c_i$$

\subsection{Why Gumbel-Softmax is Powerful}

\begin{itemize}
    \item \textbf{Differentiable:} Can be trained with gradient descent
    \textbf{Learnable:} The quantization can improve during training
    \item \textbf{Flexible:} Can blend multiple codebook entries
    \textbf{Better gradients:} More stable training than hard VQ
\end{itemize}

\subsection{Why Gumbel-Softmax is Complex}

\begin{itemize}
    \item \textbf{Temperature scheduling:} Need to carefully control the temperature parameter
    \textbf{Not truly discrete:} The output is a weighted average, not a single entry
    \item \textbf{More computation:} Need to calculate probabilities and weighted sums
    \textbf{Hyperparameter tuning:} More parameters to adjust
\end{itemize}

\section{Understanding the Audio Codec Architecture}

\subsection{The Basic Structure}

A neural audio codec has three main parts:
\begin{enumerate}
    \item \textbf{Encoder:} Converts audio into a compressed representation
    \item \textbf{Quantizer:} Makes the representation discrete (this is what we're studying)
    \item \textbf{Decoder:} Converts the compressed representation back to audio
\end{enumerate}

\textbf{The Analogy:} Like a translation service:
\begin{itemize}
    \item \textbf{Encoder:} English speaker who understands your message
    \textbf{Quantizer:} The translator who converts it to another language
    \item \textbf{Decoder:} Someone who speaks that language and converts it back
\end{itemize}

\subsection{The Encoder}

The encoder is like a smart audio analyzer:
\begin{itemize}
    \item \textbf{Input:} Raw audio samples (like 16,000 numbers for 1 second)
    \textbf{Processing:} Convolutional layers, recurrent layers, attention mechanisms
    \item \textbf{Output:} A compressed representation (like 100 numbers)
\end{itemize}

\textbf{What the Encoder Learns:}
\begin{itemize}
    \item Which audio features are important
    \item How to represent audio efficiently
    \item What patterns to look for
\end{itemize}

\subsection{The Quantizer}

The quantizer is what we're studying in this project:
\begin{itemize}
    \item \textbf{Input:} Continuous representation from encoder
    \textbf{Processing:} VQ, RVQ, PQ, or Gumbel-Softmax
    \item \textbf{Output:} Discrete, compressed representation
\end{itemize}

\textbf{Why Quantization Matters:}
\begin{itemize}
    \item \textbf{Compression:} Makes the representation much smaller
    \textbf{Discrete:} Can be stored and transmitted efficiently
    \item \textbf{Trade-off:} Smaller size vs. better quality
\end{itemize}

\subsection{The Decoder}

The decoder is like a smart audio synthesizer:
\begin{itemize}
    \item \textbf{Input:} Discrete representation from quantizer
    \textbf{Processing:} Convolutional layers, recurrent layers, upsampling
    \item \textbf{Output:} Reconstructed audio
\end{itemize}

\textbf{What the Decoder Learns:}
\begin{itemize}
    \item How to reconstruct audio from compressed representation
    \item Which details to emphasize
    \item How to make the output sound natural
\end{itemize}

\section{Understanding the Training Process}

\subsection{How Neural Codecs Learn}

Neural codecs learn through a process called "end-to-end training":
\begin{enumerate}
    \item \textbf{Forward pass:} Audio goes through encoder → quantizer → decoder
    \item \textbf{Calculate loss:} Compare original audio to reconstructed audio
    \item \textbf{Backward pass:} Calculate gradients and update all parameters
    \item \textbf{Repeat:} Do this millions of times with different audio samples
\end{enumerate}

\textbf{The Analogy:} Like learning to draw:
\begin{enumerate}
    \item \textbf{Try to draw:} Make your best attempt at drawing a cat
    \textbf{Compare:} Look at your drawing vs. a real cat photo
    \item \textbf{Learn:} Figure out what you did wrong
    \textbf{Practice:} Try again with what you learned
\end{enumerate}

\subsection{The Loss Function}

The loss function measures how different the original and reconstructed audio are:

\textbf{Common Loss Functions:}
\begin{itemize}
    \item \textbf{L1 Loss:} $\mathcal{L}_{L1} = \frac{1}{N} \sum_{i=1}^N |x_i - \hat{x}_i|$
    \textbf{L2 Loss:} $\mathcal{L}_{L2} = \frac{1}{N} \sum_{i=1}^N (x_i - \hat{x}_i)^2$
    \item \textbf{Perceptual Loss:} How different the audio "sounds" to humans
    \textbf{Adversarial Loss:} How realistic the reconstructed audio sounds
\end{itemize}

\textbf{What Each Loss Measures:}
\begin{itemize}
    \item \textbf{L1/L2:} Mathematical difference between original and reconstruction
    \textbf{Perceptual:} How similar the audio sounds to humans
    \item \textbf{Adversarial:} How well the reconstruction fools a discriminator
\end{itemize}

\subsection{The Training Challenges}

\textbf{Challenge 1: Quantization is Not Differentiable}
\begin{itemize}
    \item \textbf{Problem:} VQ makes hard decisions that can't be differentiated
    \textbf{Solutions:} Use Gumbel-Softmax, straight-through estimator, or proxy losses
\end{itemize}

\textbf{Challenge 2: Codebook Collapse}
\begin{itemize}
    \item \textbf{Problem:} Many codebook entries end up representing the same thing
    \textbf{Solutions:} Regularization, diversity constraints, or better initialization
\end{itemize}

\textbf{Challenge 3: Training Stability}
\begin{itemize}
    \item \textbf{Problem:} Training can be unstable due to quantization
    \textbf{Solutions:} Careful learning rate scheduling, gradient clipping, or warm-up
\end{itemize}

\section{Understanding the Evaluation Metrics}

\subsection{How We Measure Quality}

\textbf{Objective Metrics:} Numbers that measure audio quality mathematically
\begin{itemize}
    \item \textbf{Signal-to-Noise Ratio (SNR):} How much signal vs. noise
    \textbf{Signal-to-Distortion Ratio (SDR):} How much signal vs. distortion
    \item \textbf{Scale-Invariant SDR (SI-SDR):} SDR that's not affected by volume changes
\end{itemize}

\textbf{Subjective Metrics:} How good the audio sounds to human listeners
\begin{itemize}
    \item \textbf{Mean Opinion Score (MOS):} Average rating from human listeners
    \textbf{ABX Testing:} Can listeners tell which audio is better?
    \item \textbf{Preference Testing:} Which version do listeners prefer?
\end{itemize}

\subsection{How We Measure Compression}

\textbf{Bitrate:} How many bits we use per second of audio
\begin{itemize}
    \item \textbf{Low bitrate:} 8-32 kbps (like old cell phones)
    \textbf{Medium bitrate:} 64-128 kbps (like MP3)
    \item \textbf{High bitrate:} 256-320 kbps (like high-quality MP3)
    \textbf{Very high bitrate:} 1000+ kbps (like lossless audio)
\end{itemize}

\textbf{Compression Ratio:} How much smaller the compressed file is
\begin{itemize}
    \item \textbf{10:1 compression:} File is 10 times smaller
    \textbf{100:1 compression:} File is 100 times smaller
    \item \textbf{1000:1 compression:} File is 1000 times smaller
\end{itemize}

\subsection{How We Measure Speed}

\textbf{Encoding Time:} How long it takes to compress audio
\begin{itemize}
    \item \textbf{Real-time:} Can compress 1 second of audio in less than 1 second
    \textbf{Faster than real-time:} Can compress 1 second of audio in 0.1 seconds
    \item \textbf{Slower than real-time:} Takes 10 seconds to compress 1 second of audio
\end{itemize}

\textbf{Decoding Time:} How long it takes to decompress audio
\begin{itemize}
    \item \textbf{Real-time:} Can play back compressed audio without delays
    \textbf{Faster than real-time:} Can play back faster than normal speed
    \item \textbf{Slower than real-time:} There are delays during playback
\end{itemize}

\section{Understanding the Implementation}

\subsection{The Project Structure}

Our audio codec project is organized as follows:
\begin{itemize}
    \item \texttt{models/}: Different neural network architectures
    \item \texttt{quantizers/}: Different quantization methods (VQ, RVQ, PQ, Gumbel-Softmax)
    \item \texttt{datasets/}: Audio datasets for training and testing
    \item \texttt{training/}: Training scripts and utilities
    \item \texttt{evaluation/}: Code for measuring performance
    \item \texttt{utils/}: Helper functions and utilities
\end{itemize}

\subsection{The Main Training Script}

Here's what a typical training script looks like:

\begin{lstlisting}[language=Python, basicstyle=\small]
def train_audio_codec():
    # Load dataset
    train_loader = load_audio_dataset('train')
    val_loader = load_audio_dataset('val')
    
    # Create model
    encoder = AudioEncoder()
    quantizer = VectorQuantizer(codebook_size=512, embedding_dim=64)
    decoder = AudioDecoder()
    
    # Create optimizer
    optimizer = torch.optim.Adam([
        {'params': encoder.parameters()},
        {'params': quantizer.parameters()},
        {'params': decoder.parameters()}
    ], lr=1e-4)
    
    # Training loop
    for epoch in range(num_epochs):
        for batch in train_loader:
            # Forward pass
            audio = batch['audio']
            encoded = encoder(audio)
            quantized = quantizer(encoded)
            reconstructed = decoder(quantized)
            
            # Calculate loss
            loss = calculate_loss(audio, reconstructed)
            
            # Backward pass
            optimizer.zero_grad()
            loss.backward()
            optimizer.step()
\end{lstlisting}

\textbf{What This Does Step by Step:}
\begin{enumerate}
    \item Load training and validation data
    \item Create the encoder, quantizer, and decoder
    \item Set up the optimizer to update parameters
    \item For each training batch:
    \begin{itemize}
        \item Pass audio through the model
        \item Calculate how good the reconstruction is
        \item Update the model to do better
    \end{itemize}
\end{enumerate}

\subsection{The Quantizer Implementation}

Here's what a basic VQ quantizer looks like:

\begin{lstlisting}[language=Python, basicstyle=\small]
class VectorQuantizer(nn.Module):
    def __init__(self, codebook_size, embedding_dim):
        super().__init__()
        self.codebook_size = codebook_size
        self.embedding_dim = embedding_dim
        
        # Create the codebook
        self.embedding = nn.Embedding(codebook_size, embedding_dim)
        self.embedding.weight.data.uniform_(-1/32, 1/32)
    
    def forward(self, z):
        # z has shape (batch_size, seq_len, embedding_dim)
        batch_size, seq_len, embedding_dim = z.shape
        
        # Reshape for easier processing
        z_flat = z.view(-1, embedding_dim)
        
        # Find closest codebook entry
        distances = torch.cdist(z_flat, self.embedding.weight)
        indices = torch.argmin(distances, dim=1)
        
        # Get quantized vectors
        quantized = self.embedding(indices)
        
        # Reshape back
        quantized = quantized.view(batch_size, seq_len, embedding_dim)
        
        return quantized, indices
\end{lstlisting}

\textbf{What This Does:}
\begin{enumerate}
    \item Create a codebook of learnable vectors
    \item For each input vector, find the closest codebook entry
    \item Return the quantized vector and the index used
\end{enumerate}

\subsection{The Loss Calculation}

Here's how we calculate the training loss:

\begin{lstlisting}[language=Python, basicstyle=\small]
def calculate_loss(original, reconstructed, indices, embedding):
    # Reconstruction loss (how different the audio sounds)
    recon_loss = F.mse_loss(reconstructed, original)
    
    # Commitment loss (keep encoder output close to codebook)
    commit_loss = F.mse_loss(embedding, original.detach())
    
    # Total loss
    total_loss = recon_loss + 0.25 * commit_loss
    
    return total_loss
\end{lstlisting}

\textbf{What Each Loss Term Does:}
\begin{itemize}
    \item \textbf{Reconstruction loss:} Make the output sound like the input
    \textbf{Commitment loss:} Make the encoder output close to codebook entries
    \item \textbf{Total loss:} Weighted combination of both terms
\end{itemize}

\section{Understanding the Experiments}

\subsection{What We're Testing}

In our project, we're comparing different quantization methods:
\begin{itemize}
    \item \textbf{Baseline:} No quantization (just encoder-decoder)
    \textbf{VQ:} Basic vector quantization
    \item \textbf{RVQ:} Residual vector quantization with 2-4 stages
    \textbf{PQ:} Product quantization with different splits
    \item \textbf{Gumbel-Softmax:} Differentiable quantization
\end{itemize}

\subsection{The Speech Enhancement Task}

We're using speech enhancement as our test case:
\begin{itemize}
    \item \textbf{Input:} Noisy speech (speech with background noise)
    \textbf{Output:} Clean speech (speech with noise removed)
    \item \textbf{Goal:} See which quantization method works best for this task
\end{itemize}

\textbf{Why Speech Enhancement?}
\begin{itemize}
    \item \textbf{Simple:} Easy to understand and evaluate
    \textbf{Realistic:} Practical application that people care about
    \item \textbf{Challenging:} Tests the limits of our quantization methods
\end{itemize}

\subsection{The Datasets We'll Use}

\textbf{VCTK Dataset:} Multi-speaker English speech corpus
\begin{itemize}
    \item \textbf{Size:} 44 hours of speech from 109 speakers
    \textbf{Quality:} High-quality recordings in quiet environments
    \item \textbf{Use:} Training and testing our models
\end{itemize}

\textbf{LibriSpeech Dataset:} Large-scale audiobook dataset
\begin{itemize}
    \item \textbf{Size:} 1000 hours of speech from 2484 speakers
    \textbf{Quality:} Professional audiobook recordings
    \item \textbf{Use:} Training on larger, more diverse data
\end{itemize}

\textbf{Custom Noisy Speech:} We'll create this by adding noise to clean speech
\begin{itemize}
    \item \textbf{Method:} Add white noise, pink noise, or real-world noise
    \textbf{Control:} We know exactly what the clean version should be
    \item \textbf{Use:} Testing how well our models can remove noise
\end{itemize}

\section{Understanding the Results}

\subsection{What We'll Measure}

\textbf{Quality Metrics:}
\begin{itemize}
    \item \textbf{SI-SDR:} How much the signal quality improves
    \textbf{PESQ:} How good the speech sounds to humans
    \item \textbf{STOI:} How intelligible the speech is
\end{itemize}

\textbf{Compression Metrics:}
\begin{itemize}
    \item \textbf{Bitrate:} How many bits per second we use
    \textbf{Compression ratio:} How much smaller the files are
    \item \textbf{Model size:} How big our trained models are
\end{itemize}

\textbf{Speed Metrics:}
\begin{itemize}
    \item \textbf{Training time:} How long it takes to train
    \textbf{Inference time:} How long it takes to process new audio
    \item \textbf{Memory usage:} How much RAM/GPU memory we need
\end{itemize}

\subsection{How to Interpret Results}

\textbf{Good Results Look Like:}
\begin{itemize}
    \item \textbf{Quality:} Reconstructed audio sounds almost as good as original
    \textbf{Compression:} Files are 10-100 times smaller
    \item \textbf{Speed:} Can process audio in real-time or faster
\end{itemize}

\textbf{What to Look For:}
\begin{itemize}
    \item \textbf{Trade-offs:} Which method gives the best quality/size balance?
    \textbf{Consistency:} Does the method work well on different types of audio?
    \item \textbf{Robustness:} Does it work well with different noise levels?
\end{itemize}

\section{Common Problems and Solutions}

\subsection{Problem 1: Poor Audio Quality}

\textbf{Symptoms:}
\begin{itemize}
    \item Reconstructed audio sounds muffled or distorted
    \item High loss values that don't decrease
    \item Audio quality gets worse during training
\end{itemize}

\textbf{Solutions:}
\begin{itemize}
    \item \textbf{Check quantization:} Make sure codebook entries are diverse
    \textbf{Adjust loss weights:} Balance reconstruction vs. commitment loss
    \item \textbf{Use better loss:} Try perceptual or adversarial losses
    \textbf{Increase model capacity:} Use bigger encoder/decoder
\end{itemize}

\subsection{Problem 2: Codebook Collapse}

\textbf{Symptoms:}
\begin{itemize}
    \item Many codebook entries are identical or very similar
    \item Poor compression because codebook isn't diverse
    \item Training gets stuck at high loss values
\end{itemize}

\textbf{Solutions:}
\begin{itemize}
    \item \textbf{Regularization:} Add diversity constraints to the loss
    \textbf{Better initialization:} Initialize codebook with diverse values
    \item \textbf{Codebook updates:} Periodically reset unused codebook entries
    \textbf{Commitment loss:} Use commitment loss to prevent collapse
\end{itemize}

\subsection{Problem 3: Training Instability}

\textbf{Symptoms:}
\begin{itemize}
    \item Loss values jump around wildly
    \item Model parameters become very large or NaN
    \item Training crashes or produces poor results
\end{itemize}

\textbf{Solutions:}
\begin{itemize}
    \item \textbf{Gradient clipping:} Limit how big gradients can become
    \textbf{Learning rate:} Use smaller learning rate or learning rate scheduling
    \item \textbf{Warm-up:} Start with small learning rate and gradually increase
    \textbf{Batch normalization:} Add batch norm layers for stability
\end{itemize}

\section{How to Contribute to the Project}

\subsection{What You Should Do First}

\begin{enumerate}
    \item \textbf{Understand the basics:} Read this document thoroughly
    \textbf{Run simple experiments:} Start with basic VQ on small datasets
    \item \textbf{Analyze results:} Look at the metrics and understand what they mean
    \textbf{Try different parameters:} Experiment with codebook sizes, embedding dimensions
    \item \textbf{Document findings:} Keep notes on what works and what doesn't
\end{enumerate}

\subsection{Areas for Improvement}

\begin{itemize}
    \item \textbf{Better quantization:} Find more effective quantization methods
    \textbf{Improved loss functions:} Design losses that work better for audio
    \item \textbf{Architecture improvements:} Try different encoder/decoder designs
    \textbf{Training tricks:} Find ways to make training more stable
    \item \textbf{Evaluation:} Design better ways to measure audio quality
\end{itemize}

\subsection{Research Questions to Explore}

\begin{itemize}
    \item \textbf{Which quantization method works best for speech?} Why?
    \textbf{How does codebook size affect quality vs. compression?}
    \item \textbf{Can we predict which method will work best for a given task?}
    \textbf{How do different loss functions affect the results?}
    \item \textbf{What makes some audio samples harder to compress than others?}
\end{itemize}

\section{Understanding the Big Picture}

\subsection{Why This Research Matters}

\begin{itemize}
    \item \textbf{Better compression:} More efficient audio storage and transmission
    \textbf{Lower bandwidth:} Better audio quality over slow internet connections
    \item \textbf{Real-time applications:} Better audio in video calls, gaming, etc.
    \textbf{Edge devices:} Better audio processing on phones and smart devices
    \item \textbf{Scientific understanding:} Learn how to represent audio efficiently
\end{itemize}

\subsection{The Broader Context}

This work fits into several larger research areas:
\begin{itemize}
    \item \textbf{Neural Compression:} Using AI to compress all types of data
    \textbf{Audio Processing:} Making computers better at understanding audio
    \item \textbf{Quantization:} Converting continuous values to discrete ones
    \textbf{Representation Learning:} Finding good ways to represent complex data
    \item \textbf{End-to-End Learning:} Training entire systems together
\end{itemize}

\subsection{Your Role in the Project}

As a researcher on this project, you should:
\begin{itemize}
    \item \textbf{Understand the code:} Know what every function does
    \textbf{Run experiments:} Test different methods and parameters
    \item \textbf{Analyze results:} Figure out what the numbers mean
    \textbf{Identify patterns:} Find what makes methods work or fail
    \item \textbf{Propose improvements:} Suggest ways to make the system better
    \textbf{Document everything:} Keep detailed notes for the research team
\end{itemize}

\section{Conclusion}

\subsection{What You Should Know Now}

After reading this document, you should understand:
\begin{itemize}
    \item What neural audio codecs are and why they're important
    \item How different quantization methods work
    \item The mathematics behind each method
    \item How to implement and train these systems
    \item How to evaluate and compare different approaches
    \item How to contribute to improving the system
\end{itemize}

\subsection{Next Steps}

\begin{enumerate}
    \item \textbf{Read the code:} Go through each file line by line
    \textbf{Run experiments:} Start with simple cases and build up
    \item \textbf{Ask questions:} Don't hesitate to ask for clarification
    \textbf{Take notes:} Document everything you learn
    \item \textbf{Think critically:} Question assumptions and look for improvements
    \textbf{Contribute ideas:} Share your insights with Minh and the team
\end{enumerate}

\subsection{Remember}

\begin{itemize}
    \item \textbf{There are no stupid questions:} If something isn't clear, ask!
    \textbf{Start simple:} Don't try to understand everything at once
    \item \textbf{Practice makes perfect:} Run lots of experiments
    \textbf{Document everything:} Your notes will help others and yourself
    \item \textbf{Think like a scientist:} Form hypotheses and test them
    \textbf{Have fun:} This is cutting-edge research - enjoy it!
\end{itemize}

\section{Glossary}

\begin{description}
    \item[Audio Codec] A system for compressing and decompressing audio data
    \item[Quantization] Converting continuous values to discrete ones
    \item[Vector Quantization (VQ)] A method that represents vectors using a codebook
    \item[Residual VQ (RVQ)] Multi-stage VQ that quantizes the error at each stage
    \item[Product Quantization (PQ)] Quantizing different parts of a vector separately
    \item[Gumbel-Softmax] A differentiable approximation of discrete sampling
    \item[Codebook] A collection of reference vectors used for quantization
    \item[Bitrate] The number of bits used per second of audio
    \item[Compression Ratio] How much smaller a compressed file is compared to original
    \item[End-to-End Training] Training all parts of a system together
    \item[Loss Function] A function that measures how good a model's output is
    \item[Gradient Descent] An optimization method that uses gradients to update parameters
    \item[Neural Network] A machine learning model inspired by biological neurons
    \item[Convolutional Layer] A type of neural network layer for processing grid-like data
    \item[Recurrent Layer] A type of neural network layer with memory
    \item[Embedding] A learned representation of data in a lower-dimensional space
\end{description}

\section{References and Further Reading}

\begin{itemize}
    \item \textbf{EnCodec Paper:} \url{https://arxiv.org/pdf/2210.13438}
    \item \textbf{SoundStream Paper:} \url{https://arxiv.org/pdf/2107.03312}
    \item \textbf{VALL-E Paper:} \url{https://arxiv.org/pdf/2301.02111}
    \item \textbf{Vector Quantization Tutorial:} \url{https://www.pinecone.io/learn/vector-quantization/}
    \item \textbf{Gumbel-Softmax Explanation:} \url{https://arxiv.org/abs/1611.01144}
    \item \textbf{Audio Processing with PyTorch:} \url{https://pytorch.org/audio/stable/tutorials/audio_feature_extractions_tutorial.html}
\end{itemize}

\end{document}
